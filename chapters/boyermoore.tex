% https://www-igm.univ-mlv.fr/~lecroq/string/node14.html

\section*{Algoritmo de Boyer-Moore}
\phantomsection
\addcontentsline{toc}{section}{Algoritmo de Boyer-Moore}

\phantomsection
\subsection*{Introducción del Algoritmo de Boyer-Moore}
\addcontentsline{toc}{subsection}{Introducción del Algoritmo de Boyer-Moore}

El algoritmo de Boyer-Moore realiza su búsqueda escaneando el patrón deseado desde la posición derecha hasta la posición izquierda. Este utiliza dos heurísticas para completar su cometido, llamadas ‘Bad Character Rule’ y ‘Good Suffix Rule’.
Para ambas heurísticas el algoritmo realiza un “preprocesamiento” del patrón, donde se elaboran dos tablas de interés:
\begin{itemize}
    \item La tabla BC (Bad Character) basa su elaboración en el desplazamiento necesario para ir desde el carácter que se encuentra más a la derecha (del patrón) hasta la primera ocurrencia de los caracteres específicos que lo componen. Si 
    \item La GS (Good Suffix) se elabora a partir de las coincidencias encontradas entre los sufijos del patrón y los caracteres restantes. Nos da el desplazamiento necesario, desde la izquierda, para encontrar dichas coincidencias. Estas pueden ser exactas (el prefijo aparece exactamente igual) o aproximadas (aparece una parte del sufijo).
\end{itemize}

Cuando el algoritmo va realizando las comparaciones entre el patrón y el texto y encuentre una discordancia, la decisión entre usar el desplazamiento recomendado por la tabla BC o la tabla GS dependerá netamente de cual de las dos le ofrezca un mayor “salto” o ventaja.

\phantomsection
\subsection*{Implementación del Algoritmo de Boyer-Moore}
\addcontentsline{toc}{subsection}{Implementación del Algoritmo de boyer-Moore}

\begin{algorithm} [H]
    \caption{Algoritmo de Boyer\_Moore}\label{alg:BM}
    \begin{algorithmic} [1]
        \Procedure{Boyer-Moore}{(patron, texto)}
            \State $sizeP = len(P)$
            \State $sizeT = len(T)$
            \State $boyerMooreBadChar = [0] * 256$ \Comment{256 es el número generalmente aceptado como alfabéto}
            \For{$0 \leq i < sizeP-1$}
                \State $boyerMooreBadChar[ord(patron[i])] = sizeP - i - 1$
            \EndFor
            
            \State $suff = [0] * sizeP$
            \State $f = 0$
            \State $g = sizeP -1$
            \State $suff[sizeP -1] = sizeP$
            \For{$sizeP -2 \geq i > -1$}
                \If{$i > g \land suff[i + sizeP -1 - f] < i - g$}
                    \State $suff[i] = suff[i + sizeP -1 -f]$
                \Else
                    \If{$i < g$}
                        \State $g = i$
                    \EndIf
                    \State $f = i$
                    \While{$g \geq 0 \land P[g] == P[g + sizeP - 1 - f]$}
                        \State $g -= 1$
                    \EndWhile
                    \State $suff[i] = f - g$
                \EndIf
            \EndFor

            \State $boyerMooreGoodSuffix = [sizeP] * sizeP$
            \For{$0 \leq i < sizeP$}
                \If{$suff[i] == i + 1$}
                    \For{$0 \leq j < sizeP - 1 - i$}
                        \If{$boyerMooreGoodSuffix[j] == sizeP$}
                            \State $boyerMooreGoodSuffix[j] = sizeP - 1 - i$
                        \EndIf
                    \EndFor
                \EndIf
            \EndFor
            \For{$0 \leq i < sizeP-1$}
                \State $boyerMooreGoodSuffix[sizeP - 1 - suff[i]] = sizeP - 1 - i$
            \EndFor

            \State $i = 0$
            \algstore{bm}
    \end{algorithmic}
\end{algorithm}

\begin{algorithm} [H]
    \begin{algorithmic} [1]
        \algrestore{bm}
                \State $j = 0$

                \While{$j \leq sizeT - sizeP$}
                \State $i = sizeP - 1$
                \While{$i != -1 \land patron[i] == texto[i+j]$}
                    \State $i -= 1$
                \EndWhile
                \If{$i < 0$}
                    \State $print(j)$
                    \State $j += boyerMooreGoodSuffix[0]$
                \Else
                    \State $j += max(boyerMooreGoodSuffix[i], boyerMooreBadChar[ord(T[i+j])] - sizeP + 1 + i)$
                \EndIf
            \EndWhile
        \EndProcedure
    \end{algorithmic}
\end{algorithm}

\phantomsection
\subsection*{Análisis del Algoritmo de Boyer-Moore}
\addcontentsline{toc}{subsection}{Análisis del Algoritmo de Boyer-Moore}

\phantomsection
\subsubsection*{Paso 1}
\addcontentsline{toc}{subsubsection}{Paso 1}

\phantomsection
\subsubsection*{Paso 2}
\addcontentsline{toc}{subsubsection}{Paso 2}

\phantomsection
\subsubsection*{Paso 3}
\addcontentsline{toc}{subsubsection}{Paso 3}

\phantomsection
\subsubsection*{Paso 4}
\addcontentsline{toc}{subsubsection}{Paso 4}

\phantomsection
\subsubsection*{Paso 5}
\addcontentsline{toc}{subsubsection}{Paso 5}

\phantomsection
\subsubsection*{Paso 6}
\addcontentsline{toc}{subsubsection}{Paso 6}


% TODO
% http://doi.acm.org/10.1145/2431211.2431212
% https://www.cs.utexas.edu/users/moore/publications/fstrpos.pdf 