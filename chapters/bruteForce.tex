\phantomsection
\section*{Fuerza Bruta (Naive Algorithm)}
\addcontentsline{toc}{section}{Fuerza Bruta (Naive Algorithm)}

\phantomsection
\subsection*{Introducción al Algoritmo de Fuerza Bruta}
\addcontentsline{toc}{subsection}{Introducción al Algoritmo de Fuerza Bruta}

\phantomsection
\subsection*{Implementación del Algoritmo de Fuerza Bruta}
\addcontentsline{toc}{subsection}{Implementación del Algoritmo de Fuerza Bruta}
% def fuerza_bruta(txt, patron):
%     N = len(txt)
%     M = len(patron)
%     i = 0   # pointer into the text
%     while i <= (N - M):
%         j = 0       # pointer into the patter
%         while j < M:
%             if txt[i+j] != patron[j]:
%                 break
%             j += 1
%         if j == M:
%             return i
%         i += 1
%     return -1
% \algnewcommand\algorithmicto{\textbf{to}}
% \algrenewtext{For}[3]%
% {\algorithmicfor\ #1 \gets #2 \algorithmicto\ #3 \algorithmicdo}

\begin{algorithm} [H]
    \caption{Algoritmo de fuerza bruta}\label{alg:FB}
    \begin{algorithmic} [1]
        \Procedure{FuerzaBruta}{(texto, patron)}
            \State $n \gets \texttt{len(texto)}$
            \State $m \gets \texttt{len(patron)}$
            \For {$i \leq (n - m)$} \Comment{Despues de $n-m$ no puede ser el patron}
                \For {$j < m$} \Comment{Evalua el patron caracter por caracter}
                    \If {texto[$i + j$] $\neq$ patron[$j$]} \Comment{Evalua si los caracteres son los mismos}
                        \State break
                    \EndIf
                \EndFor
                \If {$j = m$} \Comment{Si llega al final entonces existe y devuelve la posición}
                    \State return $i$
                \EndIf
            \EndFor
            \State return -1 \Comment{Si pasa por todo y no encuentra no existe devuelva -1}
        \EndProcedure
    \end{algorithmic}
\end{algorithm}

\phantomsection
\subsection*{Análisis del Algoritmo de Fuerza Bruta}
\addcontentsline{toc}{subsection}{Análisis del Algroitmo de Fuerza Bruta}

\subsubsection*{Paso 1}
\addcontentsline{toc}{subsubsection}{Paso 1}


\subsubsection*{Paso 2}
\addcontentsline{toc}{subsubsection}{Paso 2}


\subsubsection*{Paso 3}
\addcontentsline{toc}{subsubsection}{Paso 3}


\subsubsection*{Paso 4}
\addcontentsline{toc}{subsubsection}{Paso 4}


\subsubsection*{Paso 5}
\addcontentsline{toc}{subsubsection}{Paso 5}


\subsubsection*{Paso 6}
\addcontentsline{toc}{subsubsection}{Paso 6}