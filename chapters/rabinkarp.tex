\section*{Algoritmo de Karp-Rabin (o Rabin-Karp)}
\phantomsection
\addcontentsline{toc}{section}{Algoritmo de Karp-Rabin}

\phantomsection
\subsection*{Introducción del Algoritmo de Karp-Rabin}
\addcontentsline{toc}{subsection}{Introducción del Algoritmo de Karp-Rabin}

\phantomsection
\subsection*{Implementación del Algoritmo de Karp-Rabin}
\addcontentsline{toc}{subsection}{Implementación del Algoritmo de Karp-Rabin}

\begin{algorithm} [H]
    \caption{Algoritmo de Karp-Rabin}\label{alg:KR}
    \begin{algorithmic} [1]
        \Procedure{karp\_rabin}{(patron, texto)}
            \State $n = len(patron)$
            \State $m = len(texto)$
            \State $d = 256$ \Comment{Este es el alfabeto defecto que sale en varios analisis (caracteres alfabeto inglés)}
            \State $q = 33554393$ \Comment{Cualquier número primo sirve, pero preferiblemente alto porque los pequeños solo hacen que el algoritmo corra como fuerza bruta porque más hashes concuerdan}
            \State $h = d^{m-1} mod(q)$
            \State $ValorHashPatron = 0$
            \State $ValorHashVentanaTexto = 0$
            \State $listaIndices = []$

            \For{i = 0 < n}
                \State $ValorHashPatron = (d*ValorHashPatron + patron[i]) mod(q)$ \Comment{Esta operación sirve con un abecedario normal como tabla ascii, para usarlo en Python el accesor se tiene que meter como parametro de ord()}
                \State $ValorHashVentanaTexto = (d*ValorHashVentanaTexto + texto[i]) mod(q)$
            \EndFor

            j = 0 \Comment{definirla afuera para poder usarla dentro del for sin tener que redefinirla cada vez que empiece el for otra vez}
            \For{$i = 0 \leq m-n$}
                \If{ValorHashPatron == ValorHashVentanaTexto} \Comment{Solo hacer fuerza Bruta cuando los valores hash concuerdan}
                    \For{$j = 0 < n$}
                        \If{$patron[j] != texto[i+j]$} \Comment{Si la fuerza bruta se incumple salga}
                            \State break
                        \Else
                            \State $j += 1$
                        \EndIf
                    \EndFor
                    \If{$j == n$}
                    \algstore{kr}

    \end{algorithmic}
\end{algorithm}

\begin{algorithm}[H]
    \begin{algorithmic}[1]
        \algrestore{kr}
        \State $listaIndices += [i]$ \Comment{Al llegar al final de la fuerza bruta regista el indice}
        \EndIf
    \EndIf 
    \If{$i < m - n$}
        \State $ValorHashVentanaTexto = (d * (ValorHashVentanaTexto - texto[i] * h) + texto[i + n]) mod(q)$
        \If{$ValorHashVentanaTexto < 0$} \Comment{GeeksForGeeks recomienda en caso de que den hashes negativos}
            \State $ValorHashVentanaTexto += q$
        \EndIf
    \EndIf
\EndFor
\State return listaIndices

\EndProcedure
    \end{algorithmic}
\end{algorithm}


\phantomsection
\subsection*{Análisis del Algoritmo de Karp-Rabin}
\addcontentsline{toc}{subsection}{Análisis del Algoritmo de Karp-Rabin}

\phantomsection
\subsubsection*{Paso 1}
\addcontentsline{toc}{subsubsection}{Paso 1}

\phantomsection
\subsubsection*{Paso 2}
\addcontentsline{toc}{subsubsection}{Paso 2}

\phantomsection
\subsubsection*{Paso 3}
\addcontentsline{toc}{subsubsection}{Paso 3}

\phantomsection
\subsubsection*{Paso 4}
\addcontentsline{toc}{subsubsection}{Paso 4}

\phantomsection
\subsubsection*{Paso 5}
\addcontentsline{toc}{subsubsection}{Paso 5}
El tiempo de este algoritmo es O(n+m) normalmente, pero en el peor de los casos termina siendo igual que fuerza bruta o O(nm) siendo n y m longitudes de texto y patron

\phantomsection
\subsubsection*{Paso 6}
\addcontentsline{toc}{subsubsection}{Paso 6}