% PLANTILLA APA7
% Creado por: Isaac Palma Medina
% Última actualización: 25/07/2021
% @COPYLEFT

% Fuentes consultadas (todos los derechos reservados):  
% Normas APA. (2019). Guía Normas APA. https://normas-apa.org/wp-content/uploads/Guia-Normas-APA-7ma-edicion.pdf
% Tecnológico de Costa Rica [Richmond]. (2020, 16 abril). LaTeX desde cero con Overleaf (1 de 3) [Vídeo]. YouTube. https://www.youtube.com/watch?v=kM1KvHVuaTY Weiss, D. (2021). 
% Formatting documents in APA style (7th Edition) with the apa7 LATEX class. https://ctan.math.washington.edu/tex-archive/macros/latex/contrib/apa7/apa7.pdf @COPYLEFT

%+-+-+-+-++-+-+-+-+-+-+-+-+-++-+-+-+-+-+-+-+-+-+-+-+-+-+-+-+-+-++-+-+-+-+-+-+-+-+-+

% Preámbulo
\documentclass[stu, 11pt, letterpaper, donotrepeattitle, floatsintext, natbib, helv]{apa7}
\usepackage{lipsum}
\usepackage[utf8]{inputenc}
\usepackage{comment}
\usepackage{marvosym}
\usepackage{graphicx}
\usepackage{float}
\usepackage[normalem]{ulem}
\usepackage[spanish]{babel} 
\usepackage{titling}
\usepackage{algpseudocode}
\usepackage{algorithm}
\makeatletter
\renewcommand{\ALG@name}{Algoritmo}
\makeatother
\usepackage{setspace}
%\usepackage{titling}
\let\apasubparagraph\subparagraph
\let\subparagraph\paragraph
\usepackage[compact]{titlesec}
\let\subparagraph\apasubparagraph
\usepackage{hyperref}
\selectlanguage{spanish}
\useunder{\uline}{\ul}{}
\newcommand{\myparagraph}[1]{\paragraph{#1}\mbox{}\\}
\graphicspath{{./images/}}

\titleformat{\part}{\normalfont\LARGE\bfseries}{\thetitle. \quad }{0pt}{}[{ \titlerule[0.8pt]}]
\titleformat{\section}{\normalfont\large\bfseries}{\thetitle. \quad }{0pt}{}[{ \titlerule[0.8pt]}]
\titleformat{\subsection}{\normalfont\bfseries}{}{}{}[]


\providecommand{\keywords}[1]
{
  \small	
  \textbf{\textit{Palabras Clave---}} #1
}
% Portada
% \thispagestyle{empty}
% \title{\LARGE Sobre Matching en Hileras}
% \author{\large Victor David Coto Solano \\ Diego Quirós Artiñano \\ Derek Rojas Mendoza} % (autores separados, consultar al docente)
% Manera oficial de colocar los autores:
%\author{Autor(a) I, Autor(a) II, Autor(a) III, Autor(a) X}
% \affiliation{Universidad Nacional de Costa Rica}
% \course{EIF-203: Estructuras Discretas Grupo 01-10am}
% \professor{Carlos Loria-Saenz}
% \date{}
% \duedate{Ciclo I 2022}

% \pretitle{\begin{center}
%     \vfill}
% \posttitle{\vskip2cm
%     \end{center}}

% \preauthor{\begin{center}}
% \postauthor{ \vskip2cm
%     Universidad Nacional de Costa Rica \\
%     EIF-203: Estructuras Discretas Grupo 01-10am \\ 
%     Carlos Loria-Saenz \\
%     Ciclo I 2022 \\ 
%     \\ \vfill \includegraphics[width = 0.4\textwidth]{./UNAImage/UNA.png}\\[\bigskipamount] \vfill \vfill  
     
% \end{center}}



% Palabras claves
\begin{document}
% \maketitle
% \abstract{something, something}
% \keywords


\begin{titlepage}
    \centering
    \vfill
    \LARGE Sobre Matching en Hileras\\
    \vskip2cm
    \large Victor David Coto Solano \\
    \large Diego Quirós Artiñano \\
    \large Derek Rojas Mendoza \\
    \vskip2cm
    Universidad Nacional de Costa Rica \\
    EIF-203: Estructuras Discretas Grupo 01-10am \\ 
    Carlos Loria-Saenz \\
    Ciclo I 2022 \\
    \vfill
    \includegraphics[width = 0.4\textwidth]{./UNAImage/UNA.png} \\
    \vfill
    \vfill
    % (autores separados, consultar al docente)
    % Manera oficial de colocar los autores:
    %\author{Autor(a) I, Autor(a) II, Autor(a) III, Autor(a) X}
\end{titlepage}


% Índices
\pagenumbering{roman}
    % Contenido
%-----------------------------------------------------------------------------
% \begin{abstract}
%     This is the abstract
% \end{abstract}

% \keywords{one, two, three, four}

%------------------------------------------------------------------------------------

\newpage
%--------------------------------------------------------------------------------------------------
    \addto\captionsspanish{
    \renewcommand*\contentsname{\largeÍndice}
}
\tableofcontents
\setcounter{tocdepth}{2}
\newpage
    % Figuras
\renewcommand{\listfigurename}{\largeÍndice de fíguras}
\listoffigures
\newpage
    % Tablas
\renewcommand{\listtablename}{\largeÍndice de tablas}
\listoftables
\newpage

\renewcommand{\listalgorithmname}{\largeÍndice de algoritmos}
\listofalgorithms
% \addtocontents{loa}{\def\string\figurename{Algorithm}}
\newpage

% Cuerpo
\pagenumbering{arabic}

%------------------------------------------------------------------------------------



\begin{singlespace}
\phantomsection
\part*{Introducción General}
% \addcontentsline{toc}{part}{Introducción General}

\quad En esta investigación se cubrirá el tema de string matching y se explicarán, analizarán y construirán los 4 algoritmos solicitados por el SPEC. Se dará una breve introducción y se desarrollará cada algoritmo y se le realizará el análisis de tiempo visto en clase. Se hablará y se explicará brevemente sobre el ADN. Se hablará y se explicará brevemente la enfermedad de Huntignton, esta misma fue seleccionada para realizar un ejemplo de la función de los algoritmos string matching y su importancia en la vida cotidiana, como por ejemplo; en este caso, evaluaremos analizando una cadena de ADN, si una persona padece dicha enfermedad.

%----------------------------------------------------------------------------------------------------------------------------------------------------------------------
\phantomsection
\part*{\textit{String Matching}}% -capitulo de string matching con ejemplos ilustrativos propios
% \addcontentsline{toc}{part}{String Matching}

\phantomsection
\section*{¿Qué es String Matching?}
\addcontentsline{toc}{section}{¿Qué es String Matching?}

\quad String Matching es cuando se agarra un patrón y se buscan todas las instancias de ese patrón dentro de un texto específico. String es el término que se usan para las cadenas de texto y matching la palabra en inglés que dice que dos cosas concuerdan, en este caso la hilera del patrón con la hilera del texto.

\quad String Matching tiene varios usos en la vida real. Situaciones tan variadas como bases de datos musicales \maskCitep{FaroSimone2013Teos} y a la misma vez se puede usar para detección de plagio, forense digital, checkeo de palabras al escribir, detección de intrusión y muchísimos otros casos posibles (\cite{GeeksForGeekdsSM}). Para esta investigación nos vamos a basar en secuencias de ADN como uso del String Matching.

\phantomsection
\section*{Tipos de String Matching}
\addcontentsline{toc}{section}{Tipos de String Matching}

\quad String Matching se subdivide en dos categorías: exactas y semejantes. Las dos de las categorías tienen sus propios algoritmos diferentes. En esta investigación no solo le daremos enfoque a la categoría de algoritmos de String Matching exactos, sino que nos vamos a analizar 4 algoritmos de este, los cuales serán: Fuerza Bruta (el más básico), Knuth-Morris-Pratt, Boyer-Moore y finalmente el Rabin-Karp. A cada uno se le va a dar un enfoque mayor en la siguiente parte de la investigación, pero en general lo que los destaca es que los primeros tres son con revisión de caracteres y el último utiliza hashing (que convierte uno o varios elementos de entrada a una función en otro elemento).

%----------------------------------------------------------------------------------------------------------------------------------------
\part*{Algoritmos}
% \phantomsection
% \addcontentsline{toc}{part}{Algoritmos}

\phantomsection
\section*{Introducción}
% \addcontentsline{toc}{section}{Introducción}
\quad En esta parte de la investigación se van a analizar los cuatro algoritmos previamente mencionados. Para esto se va a explicar en términos generales como sirve el programa, se va a generar un algoritmo y finalmente se va a analizar este algoritmo con los seis pasos vistos en clase.

\phantomsection
\section*{Fuerza Bruta (Naive Algorithm)}
\addcontentsline{toc}{section}{Fuerza Bruta (Naive Algorithm)}

\phantomsection
\subsection*{Introducción al Algoritmo de Fuerza Bruta}
\addcontentsline{toc}{subsection}{Introducción al Algoritmo de Fuerza Bruta}

\quad El algoritmo de fuerza bruta es un algoritmo que analiza de izquierda a derecha revisando por caracteres si un patrón existe dentro de un texto. Se puede considerar el método más básico de String Matching porque en cada carácter revisa si el patrón se cumple.

\phantomsection
\subsection*{Implementación del Algoritmo de Fuerza Bruta}
\addcontentsline{toc}{subsection}{Implementación del Algoritmo de Fuerza Bruta}
% def fuerza_bruta(txt, patron):
%     N = len(txt)
%     M = len(patron)
%     i = 0   # pointer into the text
%     while i <= (N - M):
%         j = 0       # pointer into the patter
%         while j < M:
%             if txt[i+j] != patron[j]:
%                 break
%             j += 1
%         if j == M:
%             return i
%         i += 1
%     return -1
% \algnewcommand\algorithmicto{\textbf{to}}
% \algrenewtext{For}[3]%
% {\algorithmicfor\ #1 \gets #2 \algorithmicto\ #3 \algorithmicdo}

\begin{algorithm} [H]
    \caption{Algoritmo de fuerza bruta}\label{alg:FB}
    \begin{algorithmic} [1]
        \Procedure{FuerzaBruta}{(texto, patron)}
            \State $n \gets \texttt{len(texto)}$
            \State $m \gets \texttt{len(patron)}$
            \For {$i \leq (n - m)$} \Comment{Despues de $n-m$ no puede ser el patron}
                \For {$j < m$} \Comment{Evalua el patron caracter por caracter}
                    \If {texto[$i + j$] $\neq$ patron[$j$]} \Comment{Evalua si los caracteres son los mismos}
                        \State break
                    \EndIf
                \EndFor
                \If {$j = m$} \Comment{Si llega al final entonces existe y devuelve la posición}
                    \State return $i$
                \EndIf
            \EndFor
            \State return -1 \Comment{Si pasa por todo y no encuentra no existe devuelva -1}
        \EndProcedure
    \end{algorithmic}
\end{algorithm}

\phantomsection
\subsection*{Análisis del Algoritmo de Fuerza Bruta}
\addcontentsline{toc}{subsection}{Análisis del Algroitmo de Fuerza Bruta}

\subsubsection*{Paso 1: Establecer el tamaño n de los datos}
% \addcontentsline{toc}{subsubsection}{Paso 1: Establecer el tamaño n de los datos}
Hay dos variables que determinan el tamaño de los datos, estos serían $n$ como la cantidad de caracteres del patrón y $m$ la cantidad del texto. Entonces termina siendo $len(patrón) + len(texto) = n + m$

\subsubsection*{Paso 2: Determinar las operaciones de interés}
% \addcontentsline{toc}{subsubsection}{Paso 2: Determinar las operaciones de interés}
Las operaciones de interés del algoritmo son las comparaciones de los caracteres. Estos tendrán valor de 1 cada uno.

\subsubsection*{Paso 3: Encontrar los casos base}
% \addcontentsline{toc}{subsubsection}{Paso 3: Encontrar los casos base}
Este algoritmo implica la combinación de las dos variables de tamaño un tiempo de recorrido del patrón y otro del texto.

\[T_{patrón}(0) =  0\]
\[T_{patrón}(1) = 1\]

\[T_{patrón}(n) = 1 + T_{patrón}(n-1)\]

\[T_{texto}(0) =  0\]
\[T_{texto}(1) = 1\]

\[T_{texto}(m) = 1 + T_{texto}(m-1)\]

y esto se junta para que de (porque para cada caracter del texto se evalúan todos los del patrón):
\[T_{Fuerza Bruta}(n,m) = T_{texto}(m) * T_{patrón}(n)\]

\subsubsection*{Paso 4: Evaluando la ecuación recursiva}
% \addcontentsline{toc}{subsubsection}{Paso 4: Evaluando la ecuación recursiva}

Como es una suma de unos
\[T_{patrón}(n) = 1 + 1 + 1 + 1 ... + 1 + 1\]

\[T_{patrón}(n) = n \]

\[T_{texto}(m) = 1 + 1 + 1 + 1 ... + 1 + 1\]

\[T_{texto}(m) = m \]

Con la ecuación que sacamos en el paso anterior y la evaluación de las sub-ecuaciones entonces llegamos a la conclusión de que:
\[T_{Fuerza Bruta}(n,m) = m*n\]


\subsubsection*{Paso 5: O-grande}
% \addcontentsline{toc}{subsubsection}{Paso 5: O-grande}
Con la ecuación de tiempo entonces concluimos que el O-grande del algoritmo es: $O(n*m)$

\subsubsection*{Paso 6}
% \addcontentsline{toc}{subsubsection}{Paso 6}
*instrumentación

\section*{Knuth-Morris-Pratt (KMP)}
\phantomsection
\addcontentsline{toc}{section}{Knuth-Morris-Pratt (KMP)}

\phantomsection
\subsection*{Introducción al Algoritmo de Knuth-Morris-Pratt}
\addcontentsline{toc}{subsection}{Introducción al Algoritmo de Knuth-Morris-Pratt}
\quad Es un algoritmo que hace las comparaciones de izquierda a derecha, este realiza la búsqueda usando información basada en fallos previos obtenidos del patrón, esto se hace en una fase de “procesamiento” que crea una tabla de valores sobre su propio contenido. Esta tabla se crea para ver donde podría darse la siguiente coincidencia sin buscar más de 1 vez los caracteres del texto o cadena de caracteres donde se realiza la búsqueda, el algoritmo es de tiempo O~(n + m) siendo n = el tiempo de la fase de “searching” y m = el tiempo de la fase de procesamiento.
\\
\quad La tabla de fallos se encarga de evitar que cada carácter del texto sea analizado más de 1 vez, esto lo logra comparando el patrón consigo mismo para ver que partes se repiten. Este método guarda una lista con números le indican al algoritmo cuando debe devolverse desde la posición actual una vez que el patrón no coincida con el texto.

\quad El texto y el patrón van avanzando simultáneamente mientras ambos coincidan, si una vez coinciden del todo pero la letra siguiente sigue cumpliendo con el patrón, entonces el algoritmo mueve el patrón 1 a la derecha, si no coinciden, entonces el patrón se empieza a devolver para intentar hacerlo coincidir con el texto.

\phantomsection
\subsection*{Implementación del Algoritmo de Knuth-Morris-Pratt}
\addcontentsline{toc}{subsection}{Implementación del Algoritmo de Knuth-Morris-Pratt}
\begin{algorithm} [H]
    \caption{Algoritmo de Knuth-Morris-Pratt}\label{alg:KMP}
    \begin{algorithmic} [1]
        \Procedure{knuth\_morris\_puth}{(texto, patron)}
            \State $n \gets \texttt{len(texto)}$
            \State $m \gets \texttt{len(patron)}$
            \State $resultado = False$
            \State $listaDeIndices = []$
            \State $tablaDeFallo = [0]*m$ \Comment{Tabla que se va a usar para devoluciones en los procesamientos}
            \State $i = 0$
            \State $j = 0$
            \Procedure{Procesamiento}{(patron, m, tablaDeFallo)}
                \State $longitud = 0$ \Comment{longitud del sufijo del prefijo mas largo}
                \State $i = 1$
                \While{$i < m$}
                    \If{$\texttt{patron[$i$]} == \texttt{patron[$longitud$]}$}
                        \State $longitud += 1$
                        \State $tablaDeFallo[i] = longitud$
                        \State $i += 1$
                    
                    \Else 
                        \If {$longitud != 0$} 
                            \State $longitud = tablaDeFallo[longitud-1]$
                        \Else
                            \State $tablaDeFallo[i] = 0$
                            \State $i += 1$
                        \EndIf
                    \EndIf
                \EndWhile
            \EndProcedure
            \Procedure{Búsqueda}{()}
                \State $i = 0$
                \State $j = 0$
                \While{$i < n$}
                    \If $patron[j] == texto[i]$
                        \State $i += 1$
                        \State $j += 1$
                        \EndIf
                        \If {$j == m$}
        \State $listaDeIndices += [i-j]$
        \State $resultado = True$
        \State $j = tablaDeFallo[j-1]$
        \ElsIf {$i < n \land patron[j] != text[i]$}
            \algstore{kmp}
    \end{algorithmic}
\end{algorithm}

\begin{algorithm} [H]
    \begin{algorithmic} [1]
    \algrestore{kmp}
        \If{$j != 0$}
            \State $j = tablaDeFallo[j-1]$
        \Else 
            \State $i += 1$
        \EndIf
    \EndIf
    \EndWhile
    \EndProcedure
    \State return listaDeIndices
    \EndProcedure
        
    \end{algorithmic}
\end{algorithm}



\subsection*{Análisis del Algoritmo de Knuth-Morris-Pratt}
\addcontentsline{toc}{subsection}{Análisis del Algoritmo de Knuth-Morris-Pratt}
\subsubsection*{Paso 1: Establecer el tamaño n de los datos}
\addcontentsline{toc}{subsubsection}{Paso 1: Establecer el tamaño n de los datos}
El tamaño esta dado por la cantidad de elementos en el texto * la cantidad de elementos en el patron
\[n = len(txt) * len(pat)\]


\subsubsection*{Paso 2: Determinar las operaciones de interés}
\addcontentsline{toc}{subsubsection}{Paso 2: Determinar las operaciones de interés}
Comparaciones entre Pat y Txt y construcción de lps[] \\
    Suponemos el peor caso: todas son iguales a la mas grande de todas \\
    Asumimos que la mas grande vale 1 \\


\subsubsection*{Paso 3: Establecer la relacion de recurrencia para $\space T_{KMP}(n, m)$ con $\space n = len(txt) \land m = len(Pat)$}
\addcontentsline{toc}{subsubsection}{Paso 3: Establecer la relacion de recurrencia para $\space T_{KMP}(n, m)$ con $\space n = len(txt) \land m = len(Pat)$}
Se obtuvo que: \[T_{KMP}(n, m) = m \hskip5cm                                   si n = 0\]
\[= (2 + t_{kmpS}(n-1)) + (1 + t_{kmpP}(m-1))\hskip1cm si n > 0\]
con kmpS = KMP search y kmpP = MKP Processing


\subsubsection*{Paso 4: Resolver la ecuacion de $T_{KMP}(n, m)$ eliminando la recursion}
\addcontentsline{toc}{subsubsection}{Paso 4: Resolver la ecuacion de $T_{KMP}(n, m)$ eliminando la recursion}
\[T_{kmp}(n, m) = 2n + m\]

\subsubsection*{Paso 5: Determinar el orden de crecimiento asintotico de $T_{KMP}(n, m)$}
\addcontentsline{toc}{subsubsection}{Paso 5: Determinar el orden de crecimiento asintotico de $T_{KMP}(n, m)$}
    \[T_{kmp}(n, m) \sim O(n + m)\]
    con n = el tiempo de la fase de “searching” y m = el tiempo de la fase de procesamiento.
\subsubsection*{Paso 6}
\addcontentsline{toc}{subsubsection}{Paso 6}

*Instrumentación %TODO
% https://discord.com/channels/965795707095232552/965795707736969320/990781609223540746
% https://discord.com/channels/965795707095232552/965795707736969320/987192403209388072

% https://discord.com/channels/965795707095232552/965795707736969320/987192403209388072

% https://discord.com/channels/965795707095232552/965795707736969320/987192403209388072

% https://www-igm.univ-mlv.fr/~lecroq/string/node14.html

\section*{Algoritmo de Boyer-Moore}
\phantomsection
\addcontentsline{toc}{section}{Algoritmo de Boyer-Moore}

\phantomsection
\subsection*{Introducción del Algoritmo de Boyer-Moore}
\addcontentsline{toc}{subsection}{Introducción del Algoritmo de Boyer-Moore}

\phantomsection
\subsection*{Implementación del Algoritmo de Boyer-Moore}
\addcontentsline{toc}{subsection}{Implementación del Algoritmo de boyer-Moore}

\begin{algorithm} [H]
    \caption{Algoritmo de Boyer\_Moore}\label{alg:BM}
    \begin{algorithmic} [1]
        \Procedure{Boyer-Moore}{(patron, texto)}
            \State $sizeP = len(P)$
            \State $sizeT = len(T)$
            \State $boyerMooreBadChar = [0] * 256$ \Comment{256 es el número generalmente aceptado como alfabéto}
            \For{$0 \leq i < sizeP-1$}
                \State $boyerMooreBadChar[ord(patron[i])] = sizeP - i - 1$
            \EndFor
            
            \State $suff = [0] * sizeP$
            \State $f = 0$
            \State $g = sizeP -1$
            \State $suff[sizeP -1] = sizeP$
            \For{$sizeP -2 \geq i > -1$}
                \If{$i > g \land suff[i + sizeP -1 - f] < i - g$}
                    \State $suff[i] = suff[i + sizeP -1 -f]$
                \Else
                    \If{$i < g$}
                        \State $g = i$
                    \EndIf
                    \State $f = i$
                    \While{$g \geq 0 \land P[g] == P[g + sizeP - 1 - f]$}
                        \State $g -= 1$
                    \EndWhile
                    \State $suff[i] = f - g$
                \EndIf
            \EndFor

            \State $boyerMooreGoodSuffix = [sizeP] * sizeP$
            \For{$0 \leq i < sizeP$}
                \If{$suff[i] == i + 1$}
                    \For{$0 \leq j < sizeP - 1 - i$}
                        \If{$boyerMooreGoodSuffix[j] == sizeP$}
                            \State $boyerMooreGoodSuffix[j] = sizeP - 1 - i$
                        \EndIf
                    \EndFor
                \EndIf
            \EndFor
            \For{$0 \leq i < sizeP-1$}
                \State $boyerMooreGoodSuffix[sizeP - 1 - suff[i]] = sizeP - 1 - i$
            \EndFor

            \State $i = 0$
            \algstore{bm}
    \end{algorithmic}
\end{algorithm}

\begin{algorithm} [H]
    \begin{algorithmic} [1]
        \algrestore{bm}
                \State $j = 0$

                \While{$j \leq sizeT - sizeP$}
                \State $i = sizeP - 1$
                \While{$i != -1 \land patron[i] == texto[i+j]$}
                    \State $i -= 1$
                \EndWhile
                \If{$i < 0$}
                    \State $print(j)$
                    \State $j += boyerMooreGoodSuffix[0]$
                \Else
                    \State $j += max(boyerMooreGoodSuffix[i], boyerMooreBadChar[ord(T[i+j])] - sizeP + 1 + i)$
                \EndIf
            \EndWhile
        \EndProcedure
    \end{algorithmic}
\end{algorithm}

\phantomsection
\subsection*{Análisis del Algoritmo de Boyer-Moore}
\addcontentsline{toc}{subsection}{Análisis del Algoritmo de Boyer-Moore}

\phantomsection
\subsubsection*{Paso 1}
\addcontentsline{toc}{subsubsection}{Paso 1}

\phantomsection
\subsubsection*{Paso 2}
\addcontentsline{toc}{subsubsection}{Paso 2}

\phantomsection
\subsubsection*{Paso 3}
\addcontentsline{toc}{subsubsection}{Paso 3}

\phantomsection
\subsubsection*{Paso 4}
\addcontentsline{toc}{subsubsection}{Paso 4}

\phantomsection
\subsubsection*{Paso 5}
\addcontentsline{toc}{subsubsection}{Paso 5}

\phantomsection
\subsubsection*{Paso 6}
\addcontentsline{toc}{subsubsection}{Paso 6}

\section*{Algoritmo de Karp-Rabin (o Rabin-Karp)}
\phantomsection
\addcontentsline{toc}{section}{Algoritmo de Karp-Rabin}

\phantomsection
\subsection*{Introducción del Algoritmo de Karp-Rabin}
\addcontentsline{toc}{subsection}{Introducción del Algoritmo de Karp-Rabin}
\quad El algoritmo de Karp-Rabin es el único de los que estamos analizando que usa el método de hashing. Para esto utiliza un número primo alto y una formula para sacar el hash value del patrón y de las secciones del texto que se están evaluando. Al encontrar un valor de hash que coincida entonces va a evaluar si la sección que está evaluando en el momento es igual que el patrón, carácter por carácter. Esta es la razón por la cual se utiliza un primo grande, porque al depender de residuos (primo todos los números van a salir modulo si mismo porque el primo solo es modulo 0 con si mismo) si el número evaluando es más grande que el primo entonces puede dar el mismo valor para lo que no debería.


\phantomsection
\subsection*{Implementación del Algoritmo de Karp-Rabin}
\addcontentsline{toc}{subsection}{Implementación del Algoritmo de Karp-Rabin}

\begin{algorithm} [H]
    \caption{Algoritmo de Karp-Rabin}\label{alg:KR}
    \begin{algorithmic} [1]
        \Procedure{karp\_rabin}{(patron, texto)}
            \State $n = len(patron)$
            \State $m = len(texto)$
            \State $d = 256$ \Comment{Este es el alfabeto defecto que sale en varios analisis (caracteres alfabeto inglés)}
            \State $q = 33554393$ \Comment{Cualquier número primo sirve, pero preferiblemente alto porque los pequeños solo hacen que el algoritmo corra como fuerza bruta porque más hashes concuerdan}
            \State $h = d^{m-1} mod(q)$
            \State $ValorHashPatron = 0$
            \State $ValorHashVentanaTexto = 0$
            \State $listaIndices = []$

            \For{i = 0 < n}
                \State $ValorHashPatron = (d*ValorHashPatron + patron[i]) mod(q)$ \Comment{Esta operación sirve con un abecedario normal como tabla ascii, para usarlo en Python el accesor se tiene que meter como parametro de ord()}
                \State $ValorHashVentanaTexto = (d*ValorHashVentanaTexto + texto[i]) mod(q)$
            \EndFor

            j = 0 \Comment{definirla afuera para poder usarla dentro del for sin tener que redefinirla cada vez que empiece el for otra vez}
            \For{$i = 0 \leq m-n$}
                \If{ValorHashPatron == ValorHashVentanaTexto} \Comment{Solo hacer fuerza Bruta cuando los valores hash concuerdan}
                    \For{$j = 0 < n$}
                        \If{$patron[j] != texto[i+j]$} \Comment{Si la fuerza bruta se incumple salga}
                            \State break
                        \Else
                            \State $j += 1$
                        \EndIf
                    \EndFor
                    \If{$j == n$}
                    \algstore{kr}

    \end{algorithmic}
\end{algorithm}

\begin{algorithm}[H]
    \begin{algorithmic}[1]
        \algrestore{kr}
        \State $listaIndices += [i]$ \Comment{Al llegar al final de la fuerza bruta regista el indice}
        \EndIf
    \EndIf 
    \If{$i < m - n$}
        \State $ValorHashVentanaTexto = (d * (ValorHashVentanaTexto - texto[i] * h) + texto[i + n]) mod(q)$
        \If{$ValorHashVentanaTexto < 0$} \Comment{GeeksForGeeks recomienda en caso de que den hashes negativos}
            \State $ValorHashVentanaTexto += q$
        \EndIf
    \EndIf
\EndFor
\State return listaIndices

\EndProcedure
    \end{algorithmic}
\end{algorithm}


\phantomsection
\subsection*{Análisis del Algoritmo de Karp-Rabin}
\addcontentsline{toc}{subsection}{Análisis del Algoritmo de Karp-Rabin}

\subsubsection*{Paso 1: Establecer el tamaño n de los datos}
% \addcontentsline{toc}{subsubsection}{Paso 1: Establecer el tamaño n de los datos}
Las dos variables que varían el tamaño de datos es la cantidad de caracteres del patrón $n$ y la del texto $m$. Entonces termina siendo $len(patrón) + len(texto) = n + m$

\subsubsection*{Paso 2: Determinar las operaciones de interés}
% \addcontentsline{toc}{subsubsection}{Paso 2: Determinar las operaciones de interés}
Las operaciones de interés son las comparaciones tanto del valor de hash como el carácter y van a contar como 1.

\subsubsection*{Paso 3: Encontrar los casos base}
% \addcontentsline{toc}{subsubsection}{Paso 3: Encontrar los casos base}
El algoritmo tiene tres sub-ecuaciones el de tiempo que tarda comparar los caracteres de patrón, los caracteres de texto y las comparaciones de los hashes.
\[T_{patrón}(0) =  0\]
\[T_{patrón}(1) = 1\]

\[T_{patrón}(n) = 1 + T_{patrón}(n-1)\]

\[T_{texto}(0) =  0\]
\[T_{texto}(1) = 1\]

\[T_{texto}(m) = 1 + T_{texto}(m-1)\]

\[T_{hashes}(0) = 0\]
\[T_{hashes}(1) = 1\]
\[T_{hashes}(m) = 1 + T_{hashes}(m-1)\]


\[T_{Karp-Rabin}(n,m) = T_{hashes}(m) + T_{texto}(m) * T_{patrón}(n)\]

Se hace de esta manera porque siempre se evalúan los hashes y después por aparte se hace un fuerza bruta de la sección.
\subsubsection*{Paso 4: Evaluando la ecuación recursiva}
% \addcontentsline{toc}{subsubsection}{Paso 4: Evaluando la ecuación recursiva}

\[T_{patrón}(n) = 1 + 1 + 1 + 1 ... + 1 + 1\]

\[T_{patrón}(n) = n \]

\[T_{texto}(m) = 1 + 1 + 1 + 1 ... + 1 + 1\]

\[T_{texto}(m) = m \]

\[T_{hashes}(m) = 1 + 1 + 1 + 1 ... + 1 + 1\]

\[T_{hashes}(m) = m \]

\[T_{Karp-Rabin}(n,m) = m + m * n\]

\subsubsection*{Paso 5: O-grande}
% \addcontentsline{toc}{subsubsection}{Paso 5: O-grande}

Dado a lo que evaluamos en los últimos pasos se sabe que O-grande es $O(m) + O(m*n) \sim O(m*n)$

Es importante notar que ese tiempo ocurre cuando todos los caracteres son lo mismo entonces tiene que evaluar todos los caracteres. El tiempo normal es $O(m) + O(cn) \sim O(m+n), c$ constante.


\subsubsection*{Paso 6}
% \addcontentsline{toc}{subsubsection}{Paso 6}
*instrumentación

\phantomsection
\part*{ADN}% -capitulo de string matching con ejemplos ilustrativos propios
% \addcontentsline{toc}{part}{String Matching}

\phantomsection
\section*{Importancia del String Matching con el ADN}
% \addcontentsline{toc}{section}{Importancia del String Matching con el ADN}

\quad El Ácido desoxirribonucleico o ADN es un ácido nucleico compuesto de 4 nucleótidos, estos son la adenina (A), timina (T), guanina (G) y citosina (C), además de estos 4 nucleótidos, existe un 5 que es el uracilo (U), pero este se encuentra en el ARN reemplazando a la timina (T). Este contiene las instrucciones y la información genética usadas en el desarrollo y funcionamiento de todo ser vivo, incluyendo algunos virus, además, es el responsable de la herencia genética, como el color de ojos, el color de piel, el cabello y todo tipo de cosas que podemos heredar de nuestros padres o antepasados, sin embargo, también es responsable de heredar ciertas enfermedades genéticas, como por ejemplo, la fibrosis quística, la hemofilia, la enfermedad de Huntington, etc (\cite{ADN1}). \\

\quad Con esto se buscar una base nitrogenada específica, hicimos pruebas con los cuatro algoritmos con una secuencia de ADN aleatoria generada en \cite{ADNGen}, y le corrimos con timeit.repeat() 
\begin{table} {H}
    \begin{tabular}{| c | c | c | c |}
        \hline
        Fuerza Bruta & Knuth-Morris-Pratt & Boyer-Moore & Karp-Rabin \\
        \hline
        0.21230525700957514 s & 0.21744267400936224 s & 0.3150837090215646 s & 0.36325935399509035 s \\
        \hline
    \end{tabular}
    \caption{ADN aleatorio buscando adeina, generada por \cite{ADNGen} (Apéndice pruebas.ipynb)}
    \label{tab:DNAR}
\end{table}
\quad Para efectos de esta investigación, hemos decidido utilizar la enfermedad de Huntington como ejemplo de string matching en una cadena de ADN. \\

\quad La enfermedad de Huntington es una enfermedad hereditaria que da la instrucción al cuerpo de producir una proteína llamada Huntingtina(HTT). Aunque se desconoce la función de dicha proteína, se cree que juega un papel importante en las neuronas. Esta enfermedad es provocada por una mutación en un segmento del ADN conocido como una repetición del trinucleótido CAG (citosina, adenina y guanina). Este segmento CAG normalmente se repite de 10 a 35 veces en personas sanas, sin embargo, en personas con la enfermedad de Huntington, dicho segmento se repite de 36 a as de 120 veces. \\

Como se puede ver en la siguiente imagen: 
\begin{figure} [H]
\includegraphics[width=0.6\textwidth]{causas-enfermedad-huntington.jpg}
\caption{Causas de Huntington (\cite{ADN2})}
\label{fig:Hunttington}
\end{figure}


\quad Para efectos del string matching, vamos a comprobar cuantas veces se repite  el segmento CAG en la cadena de ADN para comprobar si una persona padece de esta enfermedad.

\begin{table} [h!]
    \begin{tabular}{|c| c | c | c | c |}
        \hline
        & Fuerza Bruta & Knuth-Morris-Pratt & Boyer-Moore & Karp-Rabin \\
        \hline
        Sin Huntington & 0.0561 s & 0.0558 s& 0.089 s & 0.0815 s\\
        \hline
        Con Huntington & 0.0837 s & 0.0714 s & 0.1077 s & 0.1055 s\\
        \hline
    \end{tabular}
    \caption{Secuencia de ejemplo sacada de \cite{ADNCATSEQUENCE}, buscando Huntington (Apéndice pruebas.ipynb)}
    \label{tab:DNAH}
\end{table}

% https://biolinksperu.com/blog/enfermedades-hereditarias-comunes/
% https://rochepacientes.es/enfermedad-huntington/causas.html

%-----------------------------------------------------------------------------------------------------------------------------------------------------
\phantomsection
\section*{Conclusión}
% \addcontentsline{toc}{section}{Conclusión}
\quad De esta investigación aprendimos, investigamos, estudiamos e intentamos entender lo mejor posible sobre algunos de los algoritmos de string matching existentes. Se profundizó sobre definiciones y se explicaron los 4 algoritmos solicitados. Hicimos análisis de tiempo de los algoritmos y aprendimos acerca de la prueba experimental en el campo de la informática. Se habló sobre el ADN y la enfermedad de Huntington y se aplicó un ejemplo sobre dicha enfermedad para probar los algoritmos. Aprendimos sobre la importancia que pueden tener los algoritmos de string matching en la vida cotidiana, tanto en algo tan complejo como detectar enfermedades en cadenas de ADN como algo tan simple como buscar una palabra en un texto. En el proceso aprendimos más sobre diferentes conceptos relacionados con string matching, aprendimos más sobre Python y sobre los algoritmos solicitados. Creamos implementaciones de dichos algoritmos utilizando diferentes fuentes y seudocódigos e intentamos darles distintos toques personales. Sin embargo, consideramos que aun tenemos un largo camino por recorrer dado que algunos de los datos “teóricos” fueron muy diferentes a los prácticos, pero igualmente le dedicamos mucho esfuerzo..

%----------------------------------------------------------------------------------------------------------------------------------------------------------------
\appendix

% @online{TwoWaySM,
%     author = {Maxime Crochemore} and {Dominique Perrin},
%     title = {Two-Way String-Matching},
%     year = {1991},
%     url = {https://igm.univ-mlv.fr/~mac/Articles-PDF/CP-1991-jacm.pdf}
% }
\newpage
% Referencias
\renewcommand\refname{\large\textbf{Referencias}}
\bibliography{ref}
\end{singlespace}
\end{document}